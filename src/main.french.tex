% FortySecondsCV LaTeX template
% Copyright © 2019 René Wirnata <rene.wirnata@pandascience.net>
% Licensed under the 3-Clause BSD License. See LICENSE file for details.
%
% Attributions
% ------------
% * fortysecondscv is based on the twentysecondcv class by Carmine Spagnuolo
%   (cspagnuolo@unisa.it), released under the MIT license and available under
%   https://github.com/spagnuolocarmine/TwentySecondsCurriculumVitae-LaTex
% * further attributions are indicated immediately before corresponding code

%-------------------------------------------------------------------------------
%                             ADDITIONAL PACKAGES
%-------------------------------------------------------------------------------
\documentclass[
  a4paper,
%   showframes,
   maincolor=cvblue,
   sectioncolor=cvblue,
%  subsectioncolor  =orange
   sidebarwidth=0.323\paperwidth,
%   topbottommargin=0.03\paperheight,
%   leftrightmargin=20pt
]{fortysecondscv}

% improve word spacing and hyphenation
\usepackage{microtype}
\usepackage{ragged2e}
\usepackage{rotating}
\usepackage{enumitem}


% take care of proper font encoding
\ifxetex
	\usepackage{fontspec}
	\defaultfontfeatures{Ligatures=TeX}
% \newfontfamily\headingfont[Path = fonts/]{segoeuib.ttf} % local font
\else
	\usepackage[utf8]{inputenc}
	\usepackage[T1]{fontenc}
% \usepackage[sfdefault]{noto} % use noto google font
\fi

% enable mathematical syntax for some symbols like \varnothing
\usepackage{amssymb}

% bubble diagram configuration
\usepackage{smartdiagram}
\smartdiagramset{
  % defaut font size is \large, so adjust to harmonize with sidebar layout
  bubble center node font = \footnotesize,
  bubble node font = \footnotesize,
  % default: 4cm/2.5cm; make minimum diameter relative to sidebar size
  bubble center node size = 0.3\sidebartextwidth,
  bubble node size = 0.25\sidebartextwidth,
  distance center/other bubbles = 1.5em,
  % set center bubble color
  bubble center node color = maincolor!70,
  % define the list of colors usable in the diagram
  set color list = {maincolor!10, maincolor!40,
  maincolor!20, maincolor!60, maincolor!35},
  % sets the opacity at which the bubbles are shown
  bubble fill opacity = 0.8,
}


%-------------------------------------------------------------------------------
%                            PERSONAL INFORMATION
%-------------------------------------------------------------------------------
% profile picture
\cvprofilepic{pics/profile}
% your name
\cvname{\LARGE Hugo MACHEFER}
% job title/career
\cvjobtitle{Concepteur Full Stack  \\   | Devops | Tech Lead }

% phone number
\cvphone{001 514 659 0718}
% email address
\cvmail{hugo.machefer@gmail.com}

\cvlinkedin{linkedin.com/in/hugomachefer}{hugomachefer}

\cvgithub{https://github.com/hmachefe}{hugomachefer}

% short address/location, use \newline if more than 1 line is required
\cvaddress{Montréal, QC (Canada)}


% add additional information
% \newcommand{\additional}{some more?}


% add more profile sections to sidebar on first page
\addtofrontsidebar{
	% include gosquare national flags from https://github.com/gosquared/flags;
	% naming according to ISO 3166-1 alpha-2 country codes
	\graphicspath{{pics/flags/}}

  \profilesection{Compétences}
  \chartlabel{Langages de programmation}\\
      \pointskillNinety{}{TypeScript, JavaScript}
      \pointskillNinety{}{HTML (HTML 5), CSS (3D)}
      \pointskillEighty{}{PHP (v7 et v5)}
      \pointskillSeventy{}{Python (v2 et v3)}
      \pointskillSeventy{}{C\# (.NET, Entity Framework)}
      \pointskillFifty{}{Java (Fx, Swing, Maven)}
      \pointskillFifty{}{Matlab}
      \pointskillFortyQt{}{C++ / C   (QT 5)}
      \chartlabel{Développement Front-end (web)}\\
      \pointskillNinety{}{Angular (9/10/14/17.1)}
      \pointskillSeventy{}{React, jQuery}
      \pointskillEighty{}{RxJS / Observables}
      \pointskillEighty{}{Angular Material / PrimeNG}
      \pointskillSeventy{}{SCSS / SASS, Bootstrap}
      \pointskillSixty{}{Backbone, VueJS}
      \pointskillForty{}{ThreeJS (3D)}
      \pointskillSeventy{}{Electron}
      \chartlabel{Développement Back-end}\\
      \pointskillNinety{}{OpenAPI/Swagger (TSOA)}
      \pointskillSeventy{}{Express, Node.js}
      \pointskillSixty{}{Microservices (en Python)}
      \pointskillSixty{}{Yii (PHP framework)}
      \pointskillSixty{}{Knex (Postgres, SQLite)}
      \pointskillThirty{}{Django (Python framework)}
      \pointskillThirty{}{Elixir (Robust scalability)}
      \chartlabel{Bases de données}\\
      \pointskillNinety{}{DynamoDB / (MongoDB)}
      \pointskillEighty{}{ElasticSearch}
      \pointskillSeventy{}{Postgres, SQLite, MySQL}
      \pointskillSeventy{}{MySQL Workbench, DBeaver}
      \pointskillForty{}{Microsoft SQL Server}
    \chartlabel{Protocoles et sécurité}\\
      \pointskillEighty{}{HTTP / HTTPS / SSL}
      \pointskillSeventy{}{Let's Encrypt}
      \pointskillSeventy{}{Auth0, Auth 2, GraphQL}
}


%-------------------------------------------------------------------------------
%                              SIDEBAR 2nd PAGE
%-------------------------------------------------------------------------------
\addtobacksidebar{
\vspace*{-3mm}  % Adjusts the vertical space, move content up by 10mm
\chartlabel{Infonuagique AWS}\\
\pointskillNinety{}{EC2, S3, CloudWatch}
\pointskillNinety{}{Lambdas (layer)}
\pointskillNinety{}{Secrets Manager}
\pointskillEighty{}{IAM, ECR, SES, SNS}
\pointskillEighty{}{CloudFront (CDN)}
\pointskillSeventy{}{Elastic Beanstalk}
\pointskillSeventy{}{Step Functions}
\pointskillTwenty{}{Fargate (ECS)}
\pointskillEighty{}{Route53 (DNS)}
\pointskillEighty{}{Certificates (ACM)}

\chartlabel{Infonuagique Azure}\\
\pointskillForty{}{Text To Speech Service}

\chartlabel{DevOps}\\
\pointskillEighty{}{Terraform (IaC Multi-Cloud)}
\pointskillSixty{}{Kubernetes (Orchestration)}
\pointskillFifty{}{CloudFormation (IaC AWS)}

\chartlabel{Lecteurs HTML5}\\
\pointskillNinety{}{Bitmovin}
\pointskillNinety{}{Accurate Player (Codemill)}
\pointskillSeventy{}{Video.js (open source)}

\chartlabel{Multimédia}\\
\pointskillNinety{}{ffmpeg, mediainfo, ffprobe}
\pointskillNinety{}{AWS MediaConvert}
\pointskillSeventy{}{Hybrik (Dolby)}

\chartlabel{Développement mobile}\\
\pointskillSixty{}{Java Android (Dalvik)}
\pointskillForty{}{Ionic (Apache Cordova)}
\pointskillThirty{}{Xcode (Objective C)}

\chartlabel{Systèmes de contrôle de version}\\
\pointskillEighty{}{Git (GitLab, GitHub)}
\pointskillForty{}{SVN / Synergy}

\chartlabel{Systèmes d’exploitation}\\
\pointskillEighty{}{Linux (Ubuntu 22)}
\pointskillSeventy{}{Windows (Win 10)}
\pointskillSixty{}{MacOS (El Capitan 10.11.2)}
\pointskillSeventy{}{Tizen (Samsung TV)}
\pointskillForty{}{WebOS (LG TV)}

\chartlabel{Environnements de dév (IDE)}\\
\pointskillEighty{}{WebStorm (JetBrains)}
\pointskillEighty{}{Visual Studio Code, Cursor}
\pointskillSeventy{}{Vim / Vi}

\chartlabel{Méthodologies agiles}\\
\pointskillEighty{}{Jira (backlog, sprint)}
\pointskillEighty{}{Confluence / Wiki}
\pointskillEighty{}{Scrum (grooming, review)}

}

\addtothirdsidebar{
\vspace*{-8mm}  % Adjusts the vertical space, move content up by 10mm

\chartlabel{Scripts et automatisation}\\
\pointskillEighty{}{Bash, Shell (Linux, Mac)}
\pointskillSeventy{}{CMD prompt, PowerShell}

\chartlabel{Streaming et chiffrement DRM}\\
\pointskillEighty{}{MPEG/DASH, Widevine}
\pointskillEighty{}{HLS, AES-128}

\chartlabel{Messagerie instantanée}\\
\pointskillEighty{}{Teams, Slack}

\chartlabel{UX / UI}\\
\pointskillSeventy{}{Figma, Miro}

\chartlabel{Intelligence artificielle}\\
\pointskillNinety{}{ChatGPT 4.0}
\pointskillSixty{}{OpenAI API}

\profilesection{Langues}
\vspace{-1mm} % Ajuster l'espacement négatif si nécessaire
\barskillLang{}{\textbf{Français} (langue maternelle)}{100}
\barskillLang{}{\textbf{Anglais} (certifié TOEFL)}{92}
\barskillLang{}{\textbf{Allemand} (scolaire, 2 ans)}{35}
\barskillLang{}{\textbf{Espagnol} (débutant, Duolingo)}{25}

\profilesection{Distinctions}
\vspace{-1mm} % Ajuster l'espacement négatif si nécessaire
\skill{\faAward}{NDS CEO Award · Janvier 2006}
\skill{\faAward}{CSI Awards 2014}

\profilesection{Enseignement}
\vspace{-1mm} % Ajuster l'espacement négatif si nécessaire
\skill{\faChalkboardTeacher}{Coach HTML5 (Inde, Israël, UK, USA)}
\skill{\faChalkboardTeacher}{Formateur interculturel, TMC}
\skill{\faChalkboardTeacher}{WebRTC (Conférence, Berlin)}

\vspace{-3mm} % Ajuster l'espacement négatif si nécessaire
\profilesection{Formation}
\vspace{-1mm} % Ajuster l'espacement négatif si nécessaire
\skill{\faLaptop}{Diplôme d'ingénieur (ESEO) 1998}

\profilesection{Centres d'intérêt}
\vspace{-1mm} % Ajuster l'espacement négatif si nécessaire
\skill{\faUnity}{WebGL / VR / AR, streaming, IA}
\skill{\faGlobe}{Géopolitique internationale}
\skill{\faSwimmer}{Natation, basket-ball, cyclisme}

\profilesection{Biographie}
\vspace{-1mm} % Ajuster l'espacement négatif si nécessaire
\aboutme{
  Plus de 20 ans d'expérience en développement web, projets en télévision numérique et médias, enrichissant les architectures logicielles avec méthodologies agiles et technologies comme AWS, Angular, Vue.js, React, PHP, Python, HTML5 et IA.
}\\

}
%-------------------------------------------------------------------------------
%                         TABLE ENTRIES RIGHT COLUMN
%-------------------------------------------------------------------------------
\begin{document}


\newpage
\restoregeometry
\sidebarwidth=0.35\paperwidth

\makefrontsidebar

\vspace*{-3.7em} % Adjust the value as needed

\cvsection{Expérience professionnelle}


\begin{cvtableNew}


  \cvitemRightNew
    {Jan, 2024 – now} % Date
    {Senior Full-stack Developer Level 3} % Job title
    {Evident Scientific} % Company
    {
      \vspace{0.1pt} % Added space before "Role"
      \fontsize{10.8pt}{12pt}\selectfont % Subtle font increase
      \textbf{Role}: Application full-stack (cloud et desktop) d’inspections par boroscope\vspace{4pt}
      \begin{itemize}[itemsep=-1mm, topsep=0pt, leftmargin=8pt]
        \item Frontend: HTML, SCSS, TypeScript, Angular 17+, RxJS, primeNg, Electron\vspace{5pt}
        \item Backend: C\#, .NET 8, Entity Framework SQL, Azure Blob Storage\vspace{5pt}
        \item Design and develop APIs and microservices as Azure Function App\vspace{5pt}
      \end{itemize}
    }
    {Montreal} % Location on the second line, aligned to the right

    \vspace{1.6mm} % Add vertical space after this entry


  \cvitemRightNew
  {Janvier 2020 – aujourd’hui} % Date
  {Développeur Full Stack (DevOps) | Tech Lead} % Job title
  {difuze inc} % Company
  {
    \vspace{0.1pt}
    \fontsize{10.8pt}{12pt}\selectfont
    \textbf{Rôle} : Projets similaires à LVL Studio :\vspace{5pt}

    \begin{itemize}[itemsep=-1mm, topsep=0pt, leftmargin=8pt]
      \item \textbf{difuzego.com :} Mise à niveau Angular 14, implémentation MFA, aide à la migration ElasticSearch vers PostgreSQL (modèles Knex). Module de téléchargement via AWS Step Functions pour proxies compressés. Optimisation des workflows avec Node.js (Express) et SQLite sur site. Travail sur encodage MPEG/DASH (AWS Mediaconvert, Hybrik), HLS (AWS EC2 MacOS, FFMPEG), chiffrement AES-128. Techs: CanvasJS, bandit, Bitmovin, VideoJs, Accurate Player, compression, chiffrement.\vspace{5pt}
      \item \textbf{difuzevox.com :} Azure Text-to-Speech, API Auphonic (mixage), traitement local avec ffmpeg, mediaInfo. Génération de voix à partir de ttml, lecture multimédia par Videojs. Infra Terraform. Management 3 devs + 1 QA.\vspace{5pt}
      \item \textbf{Projet IA :} Développment de solutions pour détecter les formes de bouche ("moue"), changements de scène, et texte incrusté à l'aide d'outils IA tiers. Gestion de la communication multi-fenêtres/navigateurs (API postMessage).\vspace{5pt}
    \end{itemize}

    \textbf{CI/CD :} Déploiements Kubernetes sur site (on-premises).\vspace{5pt}

    \textbf{Modules NPM:} axios, knex, objection, pg, qrcode, speakeasy, srt2vtt, videojs...
  }
  {Montréal} % Location

  \vspace{1.6mm}



  \cvitemRightNew
    {Sep, 2017 – Jan, 2020} % Date
    {Développeur Full-Stack, gestion des actifs médias} % Job title
    {LVL Studio} % Company (without location here)
      {
        \vspace{0.2pt} % Add space before "Role"
        \fontsize{10.8pt}{12pt}\selectfont % Subtle font increase
        \textbf{Role}: En tant que contributeur majeur à la nouvelle stack logicielle de Difuze, j’ai participé à définir l’architecture et à développer les modules principaux. J’ai également géré le backlog Jira, réalisé des revues de code sur GitLab et documenté les processus dans Confluence pour la plateforme B2B difuzego.com.\vspace{5pt}

        \vspace{0.1pt} % Add space before "Projects"
        \textbf{Projets:}
        \begin{itemize}[itemsep=1mm, topsep=5pt, leftmargin=8pt]
          \item \textbf{showcase.com}: Ancienne pile YII en PHP, avec scripts C\# pour automatisation (cron jobs).
          \item \textbf{difuzego.com}: Nouvelle pile basée sur des services AWS, écrite en TS.
        \end{itemize}

        \vspace{4pt} % Add space before "Purpose"
        \textbf{Objectif}: Développement basé sur un système de gestion des ressources médias (MAM), accessible via HTTPS et le protocole SOAP, permettant l’affichage du contenu en catalogue, ainsi que le partage de films en screeners ou liens promotionnels, en amont des livraisons sous forme de commandes.\vspace{5pt}

        \textbf{Front-end}:  Angular 2 (v7.2.9), RxJS, Angular Material, ainsi que divers modules NPM t.q. ngx-translate, xml2js, ng2-file-upload, moment, socket.io etc\vspace{5pt}

        \textbf{Back-end}: Services RESTful avec OpenAPI, TypeScript Oriented Architecture (tsoa), imports NPM t.q. exceljs, fs-extra, mustache, winston, user-
        agent etc\vspace{5pt}

        \textbf{Gestion des paquets}: Bonne maîtrise de WebPack, Node.js, nvm et yarn.\vspace{5pt}

        \textbf{Base de données et recherche}: Gestion à la fois de l’ancienne pile (MySQL RDS) et de la nouvelle (DynamoDB, ElasticSearch).\vspace{5pt}

        \textbf{AWS DevOps}: Expérience avec Elastic Beanstalk, EC2, S3, Lambda Layers, CloudWatch, IAM, ECR, SES, SNS, VPC. Infra déployée avec CloudFormation.\vspace{5pt}

        \textbf{Travail additionnel}: Maintenance ancienne pile PHP/YII; dévs C\# en Mono Linux\vspace{5pt}

        \textbf{Architecture}: Architecture monolithique, modèle de données partagé UI/API, dépôt GitLab unique pour front et back.\vspace{5pt}

        \textbf{CI/CD}: Architecture monolithique, sans microservices. Modèle de données partagé entre UI et API. Dépôt GitLab commun pour front-end et back-end.
      }
    {Montreal} % Location on the second line, aligned to the right
\end{cvtableNew}








\newpage
\makebacksidebar



\begin{cvtableNew}
  \cvitemRightNew
    {Jan, 2017 – June, 2017} % Date
    {Développeur Full Stack dans le secteur de la santé} % Job title
    {AlayaCare} % Company
    {
      \vspace{0.1pt} % Add space before "Role"
      \fontsize{10.8pt}{12pt}\selectfont % Subtle font increase
      \textbf{Rôle}: Développement full-stack de fonctions web pour plateformes de santé.\par
      \vspace{4pt}
      \textbf{Front-end:} Création de composants VUE.js sans jQuery, en utilisant des bibliothèques comme super-agent pour les requêtes AJAX, moment.js pour le formatage des dates, simple-vue-validator pour la validation, et LESS pour un CSS optimisé.\par
      \vspace{4pt}
      \textbf{Back-end:} Implémentation de solutions en PHP 5 avec le framework YII, intégrant l’extension Carbon pour une meilleure gestion des dates, ainsi que des microservices Python avec SQLAlchemy. Assurance qualité par des tests unitaires rigoureux et documentation REST précise via APIARY.\par
      \vspace{4pt}
      \textbf{Mobilité et agilité des processus:} Réalisé des revues de code Android détaillées et travaillé en cycles de sprints de trois semaines, avec participation active aux stand-ups, démos, rétrospectives et séances de grooming. Gestion des tâches et du backlog via Jira, avec documentation centralisée dans Confluence.\par
      \vspace{4pt}
      \textbf{Virtualisation:} déploiement de microservices et de composants logiciels sous forme de conteneurs Docker.\par
    }
    {Montreal} % Location on the second line, aligned to the right

\vspace{2.2mm} % Add vertical space after this entry

  \cvitemRightNew
    {May, 2015 – Dec, 2016} % Date
    {Développeur Front-End — TV numérique.} % Job title
    {Abbeal for Canal+} % Company
    {
      \vspace{-0.4pt} % Add space before "Role"
      \fontsize{10.8pt}{12pt}\selectfont % Subtle font increase
      \textbf{Rôle}: Développement et maintenance WebApp et back-end pour les décodeurs Canal+ (TNT, satellite, IPTV), millions d’utilisateurs.\par
      \vspace{4pt}
      \textbf{Front-End:} Conception de solutions HTML, JS, CSS gérées par Webpack, exécutées via Qt/WebKit sous Linux. Utilisation des standards ES2015 avec Babel et Grunt.\par
      \vspace{4pt}
      \textbf{Côté serveur:} support assuré par un serveur Python UWSGI hébergeant la WebApp et gérant le routage via DBUS pour les interactions système.\par
      \vspace{4pt}
      \textbf{Bibliothèques et frameworks:} Transition de BACKBONE vers ANGULAR (MVC amélioré), avec LODASH (données), EASELJS (canvas) et EJS (templating).\par
      \vspace{4pt}
      \textbf{Méthodologies agiles et tests:} Tests unitaires avec MOCHA, SPY, SINON; tests API avec POSTMAN. Sprints agiles, suivi sur Jira, documentation sur Confluence.\par
      \vspace{4pt}
      \textbf{Déploiement:} Déployé avec succès sur 400 000 récepteurs TNT, puis étendu à plus de 2 millions d’abonnés SAT dans le monde.\par
      \vspace{4pt}
      \textbf{Projet Nightwatch:} Pilotage d’un projet de tests UI (Selenium) et simulation d’API REST, déployé sur AWS EC2 via Docker.\par
      \vspace{4pt} % Ensures there is space after the last section before any text that might follow or end of the section.
    }
    {Paris (FR)} % Location on the second line, aligned to the right


    \vspace{1.18mm} % Add vertical space after this entry

  \cvitemRightNew
    {Dec, 2014 – Apr, 2015} % Date
    {Développeur mobile et back-end | Android \& Node.js} % Job title
    {Cisco} % Company
    {
      \vspace{1pt} % Add space before "Project"
      \fontsize{10.8pt}{12pt}\selectfont % Subtle font increase
      \textbf{Projet}: Intégration de l’IoT avec la télévision avancée pour la domotique.\par
      \vspace{4pt}
      \textbf{Technologies}: IoT avec openHAB, Bluetooth (iBeacons), WiFi (Cisco MSE/Prime), RFID (Impinj). Personnalisation des chaînes TV et ciblage publicitaire via Cisco Analytics. Développement d’apps pour wearables et mobiles Samsung, serveurs webSockets et Node.js.\par
      \vspace{4pt}
      \textbf{Ambilight \& UI}: Effets ambilight développés avec GStreamer, synchronisés avec l’éclairage Philips Hue via Zigbee. Migration de l’interface de Dart/WebGL vers HTML/CSS sur la plateforme JUNO de Cisco.\par
      \vspace{4pt}
      \textbf{Distinctions}: Présenté au CES 2015, au Mobile World Congress, primé "Best IoT Product" par CSI Magazine.\par
      \vspace{4pt} % Ensures there is space after the last section before any text that might follow or end of the section.
    }
    {Paris (FR)} % Location on the second line, aligned to the right

    \vspace{1.16mm} % Add vertical space after this entry

    \cvitemRightNew
    {Aug, 2014 – Dec, 2014} % Date
    {Développeur Front-End | HTML \& JS | Applis TV num} % Job title
    {Cisco} % Company
    {
      \vspace{1pt} % Add space before "Project"
      \fontsize{10.8pt}{12pt}\selectfont % Subtle font increase
      \textbf{Projet}: Preuve de concept de WebApp pour ZON TV Portugal.\par
      \vspace{4pt}
      \textbf{Technologies}: Vanilla JS, CSS3, lecteur vidéo HTML5 dans un gestionnaire UI personnalisé pour structurer les vues MVC. Développement de composants clés : liste des chaînes, bandeau d’information, grille EPG.\par
      \vspace{4pt}
      \textbf{Intégration et flux de travail}: Utilisation de QT sur middleware RDK pour une gestion optimisée des données entre le STB et le backend. Grunt (Node) et Jenkins pour l’intégration continue, gestion des tâches via JIRA et promotion du travail d’équipe en mode Scrum.\par
      \vspace{4pt}
      \textbf{Résultat}: Livraison réussie d’une WebApp performante, améliorant l’interaction utilisateur entre le STB et le backend ZON TV, présentée comme preuve de concept lors de grands événements du secteur.
    }
    {Paris (FR)} % Location on the second line, aligned to the right
\end{cvtableNew}




\newpage
\makethirdsidebar

\vspace*{-1em} % Adjust the value as needed
\begin{cvtableNew}
  \cvitemRightNew
    {Jan, 2014 – Jul, 2014} % Date
    {Développeur Front-End | BackboneJS \& ThreeJS} % Job title
    {Cisco} % Company
    {
      \vspace{1pt}
      \fontsize{10.8pt}{12pt}\selectfont % Subtle font increase
      \textbf{Projet}: Application HTML5 pour Belgacom sur middleware RDK ; POC performant avec accélération matérielle (GPU).\par
      \vspace{4pt}
      \textbf{Responsabilités}: App basée sur CSS3D/Three.JS dans les vues BackboneJS ; AJAX, WebSockets, Hammer.JS pour tablette; empaquetée via PhoneGap.\par
    }
    {Paris (FR)} % Location on the second line, aligned to the right

    \vspace{1.57mm} % Add vertical space after this entry


  \cvitemRightNew
    {Mar, 2013 – Jan, 2014} % Date
    {Développeur CSS3D/HTML5/JS embarqué} % Job title
    {NDS} % Company
    {
      \vspace{1pt}
      \fontsize{10.8pt}{12pt}\selectfont % Subtle font increase
      \textbf{Projet}: Dév HTML5 pour smart TVs, clés HDMI, Raspberry Pi, tablettes.\par
      \vspace{4pt}
      \textbf{Responsabilités}: Fonctionnalités en HTML5, CSS3D, WebGL, BackboneJS, ThreeJS (Collada), Tween.js. Optimisation HW pour QT5/WebKit, WebKitNix, Chromium. Cross-compilation de moteurs web et gstreamers pour LG TV, Raspberry Pi, Cotton Candy, MacOS (même WebApp).\par
      \vspace{4pt}
      \textbf{Résultat}: Solutions performantes retenues par Belgacom \& Zon TV.\par
    }
    {Paris (FR)} % Location on the second line, aligned to the right

    \vspace{1.57mm} % Add vertical space after this entry

  \cvitemRightNew
    {Aug, 2012 – Feb, 2013} % Date
    {Front-End | AngularJS \& Backbone | TV num} % Job title
    {Cisco (ex NDS)} % Company
    {
      \vspace{1pt}
      \fontsize{10.8pt}{12pt}\selectfont % Subtle font increase
      \textbf{Projet}: Solution HTML5 TV numérique, UI distante brevetée, second écran.\par
      \vspace{4pt}
      \textbf{Responsabilités}: Dév d’une solution de rendu HTML5 brevetée pour la payTV, présentée à l’IBC. Évaluation de frameworks nouvelle génération (Angular, Backbone, Spine, contrôleur Google) sur STB avec benchmarks. Architecture MVC pour apps STB, étude des migrations Flash vers HTML5 (ex. Edge).\par
      \vspace{4pt}
      \textbf{Compétences}: Dév HTML5, évaluation de frameworks, architecture MVC, outils open source (GIT, Jira, Jenkins, Confluence).\par
    }
    {Paris (FR)} % Location on the second line, aligned to the right

    \vspace{1.57mm} % Add vertical space after this entry

  \cvitemRightNew
    {Dec, 2010 – Jul, 2012} % Date
    {Dév systèmes embarqués | WebKit/QT \& HTML5} % Job title
    {NDS} % Company
    {
      \vspace{1pt}
      \fontsize{10.8pt}{12pt}\selectfont % Subtle font increase
      \textbf{Projet}: Intégration QT/WebKit sur puces spécialisées, perfos et conformité.\par
      \vspace{4pt}
      \textbf{Responsabilités}: Cross-compilation de QT/WebKit pour divers chipsets STB. Passerelles Java-QT pour QWebView (Android, Qt). Distribution binaire automatisée via scripts shell (IN, IL), packaging HTML5 en .CRX pour store d’entreprise.\par
    }
    {Paris (FR)} % Location on the second line, aligned to the right

    \vspace{1.57mm} % Add vertical space after this entry

  \cvitemRightNew
  {Feb, 2002 – Nov, 2010} % Date
  {Intégration moteur web, dév widgets HTML} % Job title
  {Canal+} % Company
  {
    \vspace{1pt}
    \fontsize{10.8pt}{12pt}\selectfont % Subtle font increase
    \textbf{Projet}: Intégr$\theta$ des moteurs web Opera 6 puis NetFront dans les STB, avec fonctions avancées TV numérique. Collaboration interservices à l’échelle mondiale.\par
    \vspace{4pt}
    \textbf{Responsabilités}~: Coordination NDS/ACCESS pour l’intégration middleware STB : système fichiers, mémoire, graphisme, réseau, SSL, portage. Passerelles Java-Opera/Java-QT pour plugins type Netscape. Frameworks pour jeux STB, mise à jour HTML, optimisation multi-plateformes. Banque de widgets TV, interaction STB-iPhone, réseaux sociaux (Facebook).\par
    \vspace{4pt}
    \textbf{Compétences}: Intégration navigateur. C/C++, Java. HBBTV, CE-HTML, CSS, JS\par
  }
    {Paris (FR)} % Location on the second line, aligned to the right

    \vspace{1.57mm} % Add vertical space after this entry

  \cvitemRightNew
    {Sep, 1999 – Dec, 2001} % Date
    {Intégrateur Pay-TV, réseau de diffusion de contenu} % Job title
    {Canal+} % Company
    {
      \vspace{1pt}
      \fontsize{10.8pt}{12pt}\selectfont % Subtle font increase
      \textbf{Projet}: Intégration système et test des plateformes Canal+ à l’international.\par
      \vspace{4pt}
      \textbf{Responsabilités}: Développement de scripts de test, intégration HW/SW de bout en bout, support sur site (Sacramento, USA) et à distance pour l’Espagne, les Pays-Bas et le Japon.\par
      \vspace{4pt}
      \textbf{Aptitude}: Digital TV standards: DVB, MHP. Broadcasting: DVB-C, DVB-S, DVB-T.\par
    }
    {Paris (FR)} % Location on the second line, aligned to the right

    \vspace{1.87mm} % Add vertical space after this entry

    \cvitemRightNew
    {Oct, 1998 – Aug, 1999} % Date
    {Ingénieur DSP | Annulation d’écho acoustique} % Job title
    {ALTEN} % Company
    {
      \vspace{1pt}
      \fontsize{10.8pt}{12pt}\selectfont % Subtle font increase
      \textbf{Projet}: Pilotage du développement de solutions AEC (Annulation d’écho acoustique), améliorant la clarté audio et le mode haut-parleur sur mobile, via des techniques avancées en MATLAB et C++.\par
      \vspace{3.6pt}
      \textbf{Responsabilités}: Réglage des algos AEC pour suppression d’écho et réduction de bruit. Simulation et tests d’appels mains libres via un exécutable Visual C++.
    }
    {Issy Les Moulineaux (FR)} % Location on the second line, aligned to the right


\end{cvtableNew}




\end{document}
