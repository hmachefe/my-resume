% FortySecondsCV LaTeX template
% Copyright © 2019 René Wirnata <rene.wirnata@pandascience.net>
% Licensed under the 3-Clause BSD License. See LICENSE file for details.
%
% Attributions
% ------------
% * fortysecondscv is based on the twentysecondcv class by Carmine Spagnuolo
%   (cspagnuolo@unisa.it), released under the MIT license and available under
%   https://github.com/spagnuolocarmine/TwentySecondsCurriculumVitae-LaTex
% * further attributions are indicated immediately before corresponding code

%-------------------------------------------------------------------------------
%                             ADDITIONAL PACKAGES
%-------------------------------------------------------------------------------
\documentclass[
  a4paper,
%   showframes,
   maincolor=cvblue,
   sectioncolor=cvblue,
%  subsectioncolor  =orange
%   sidebarwidth=0.4\paperwidth,
%   topbottommargin=0.03\paperheight,
%   leftrightmargin=20pt
]{fortysecondscv}

% improve word spacing and hyphenation
\usepackage{microtype}
\usepackage{ragged2e}
\usepackage{rotating}

% take care of proper font encoding
\ifxetex
	\usepackage{fontspec}
	\defaultfontfeatures{Ligatures=TeX}
% \newfontfamily\headingfont[Path = fonts/]{segoeuib.ttf} % local font
\else
	\usepackage[utf8]{inputenc}
	\usepackage[T1]{fontenc}
% \usepackage[sfdefault]{noto} % use noto google font
\fi

% enable mathematical syntax for some symbols like \varnothing
\usepackage{amssymb}

% bubble diagram configuration
\usepackage{smartdiagram}
\smartdiagramset{
  % defaut font size is \large, so adjust to harmonize with sidebar layout
  bubble center node font = \footnotesize,
  bubble node font = \footnotesize,
  % default: 4cm/2.5cm; make minimum diameter relative to sidebar size
  bubble center node size = 0.3\sidebartextwidth,
  bubble node size = 0.25\sidebartextwidth,
  distance center/other bubbles = 1.5em,
  % set center bubble color
  bubble center node color = maincolor!70,
  % define the list of colors usable in the diagram
  set color list = {maincolor!10, maincolor!40,
  maincolor!20, maincolor!60, maincolor!35},
  % sets the opacity at which the bubbles are shown
  bubble fill opacity = 0.8,
}


%-------------------------------------------------------------------------------
%                            PERSONAL INFORMATION
%-------------------------------------------------------------------------------
% profile picture
\cvprofilepic{pics/profile}
% your name
\cvname{\LARGE Hugo MACHEFER}
% job title/career
\cvjobtitle{Full Stack Developer | Devops | Tech Lead }

% phone number
\cvphone{001 514 659 0718}
% email address
\cvmail{hugo.machefer@gmail.com}

\cvlinkedin{linkedin.com/in/hugomachefer}{hugomachefer}

\cvgithub{https://github.com/hmachefe}{hugomachefer}

% short address/location, use \newline if more than 1 line is required
\cvaddress{5563 avenue de Darlington, H3T1T1 Montréal}


% add additional information
% \newcommand{\additional}{some more?}


% add more profile sections to sidebar on first page
\addtofrontsidebar{
	% include gosquare national flags from https://github.com/gosquared/flags;
	% naming according to ISO 3166-1 alpha-2 country codes
	\graphicspath{{pics/flags/}}

	\profilesection{Skills}
    \chartlabel{Programmatic languages:}\\
      \pointskillNinety{}{TypeScript, JavaScript}
      \pointskillNinety{}{HTML (HTML 5), CSS (3D)}
      \pointskillEighty{}{PHP (v7 et v5)}
      \pointskillSeventy{}{Python (v2 et v3)}
      \pointskillSixty{}{C\# (.NET Framework, Mono)}
      \pointskillFifty{}{Java (Fx, Swing, Maven)}
      \pointskillFifty{}{Matlab}
      \pointskillForty{}{QT (4 and 5)}
    \chartlabel{Front-end (web):}\\
      \pointskillNinety{}{Angular (9/10/14)}
      \pointskillEighty{}{jQuery}
      \pointskillSeventy{}{React}
      \pointskillSeventy{}{RxJS / Observables}
      \pointskillSeventy{}{SCSS / SASS}
      \pointskillSeventy{}{Bootstrap (Responsive)}
      \pointskillSixty{}{Backbone}
      \pointskillFifty{}{VueJS}
      \pointskillForty{}{ThreeJS (3D)}
      \pointskillThirty{}{LaTeX (Typesetting)}
    \chartlabel{Back-end:}\\
      \pointskillNinety{}{OpenAPI/Swagger (TSOA)}
      \pointskillSeventy{}{Express (Node.js)}
      \pointskillSixty{}{Microservices (in Python)}
      \pointskillSixty{}{Yii (PHP framework)}
      \pointskillThirty{}{Django (Python framework)}
      \pointskillThirty{}{Elixir (Robust scalability)}
      \pointskillThirty{}{Knex (Postgres, SQLite)}
    \chartlabel{Database:}\\
      \pointskillNinety{}{DynamoDB / (MongoDB)}
      \pointskillEighty{}{ElasticSearch}
      \pointskillSeventy{}{PostGres, Sqlite, MySql}
      \pointskillForty{}{Microsoft SQL Server}
      \pointskillSeventy{}{MySQLWorkbench, DBeaver}


}


%-------------------------------------------------------------------------------
%                              SIDEBAR 2nd PAGE
%-------------------------------------------------------------------------------
\addtobacksidebar{
\vspace*{-6mm}  % Adjusts the vertical space, move content up by 10mm
\chartlabel{AWS Cloud:}\\
\pointskillNinety{}{EC2, S3, CloudWatch}
\pointskillNinety{}{Lambdas (layer)}
\pointskillNinety{}{Secrets Manager}
\pointskillEighty{}{IAM, ECR, SES, SNS}
\pointskillEighty{}{CloudFront (CDN)}
\pointskillSeventy{}{Elastic Beanstalk}
\pointskillSeventy{}{Step Functions}
\pointskillTwenty{}{Fargate (ECS)}
\pointskillEighty{}{Route53 (DNS)}
\pointskillEighty{}{Certificates (ACM)}

\chartlabel{Azure Cloud:}\\
\pointskillForty{}{Text To Speech Service}

\chartlabel{DevOps:}\\
\pointskillEighty{}{Terraform (IaC Multi-Cloud)}
\pointskillSeventy{}{Kubernetes (Orchestration)}
\pointskillFifty{}{CloudFormation (IaC AWS)}

\chartlabel{HTML5 players:}\\
\pointskillNinety{}{Bitmovin}
\pointskillNinety{}{Accurate Player (Codemill)}
\pointskillSeventy{}{Video.js (open source)}

\chartlabel{Multimedia:}\\
\pointskillNinety{}{FFMPEG, mediainfo}
\pointskillNinety{}{AWS MediaConvert}
\pointskillSeventy{}{Hybrik (Dolby)}

\chartlabel{Mobile Development:}\\
\pointskillSixty{}{Java Android (Dalvik)}
\pointskillForty{}{Ionic (Apache Cordova)}
\pointskillThirty{}{Xcode (Objective C)}


\chartlabel{Version Control Systems:}\\
\pointskillEighty{}{Git (gitlab, github)}
\pointskillForty{}{SVN / Synergy}

\chartlabel{Operating Systems:}\\
\pointskillEighty{}{Linux (Ubuntu 22)}
\pointskillSeventy{}{Windows (Win 10)}
\pointskillFifty{}{Mac (El Capitan 10.11.2)}

\chartlabel{IDE:}\\
\pointskillNinety{}{Webstorm (Jetbrains)}
\pointskillEighty{}{Visual Code Editor}
\pointskillSeventy{}{Vim / Vi}


\chartlabel{Agility:}\\
\pointskillEighty{}{Jira (backlog, sprint)}
\pointskillEighty{}{Confluence / wiki}
\pointskillEighty{}{Scrum (grooming, review)}


\chartlabel{Scripting:}\\
\pointskillEighty{}{Bash, shell (Linux, Mac)}
\pointskillSeventy{}{CMD prompt, Powershell}
}

\addtothirdsidebar{
\vspace*{-9mm}  % Adjusts the vertical space, move content up by 10mm

  \chartlabel{Instant messaging:}\\
  \pointskillEighty{}{Teams, Slack}

  \chartlabel{UX / UI:}\\
  \pointskillSeventy{}{Figma, Miro}

  \chartlabel{AI:}\\
  \pointskillEighty{}{Chatgpt 4.0}
  \pointskillSixty{}{OpenAI API}

	\profilesection{Languages}
	\barskill{}{\textbf{French} (Mother Tongue)}{100}
	\barskill{}{\textbf{English} (TOEFL)}{95}
	\barskill{}{\textbf{German} (Educational, 2 years)}{35}
	\barskill{}{\textbf{Spanish} (Beginner, Duolingo)}{25}

	\profilesection{Honors \& awards}
		\skill{\faAward}{NDS CEO Award · Jan 2006}
		\skill{\faAward}{CSI Awards 2014}

  \profilesection{Teaching, Talks }
		\skill{\faChalkboardTeacher}{HTML5 coach (India, Israel, UK, US)}
    \skill{\faChalkboardTeacher}{Cross-cultural trainer, TMC}
		\skill{\faChalkboardTeacher}{ WebRTC (Intl dev conference, Berlin)}

\vspace{-3mm} % Adjust the negative space as needed
	\profilesection{Education}
		\skill{\faLaptop}{Engineer's degree (ESEO) 1998}

	\profilesection{Interests}
		\skill{\faUnity}{WebGl/VR/AR, streaming, AI}
		\skill{\faGlobe}{International geopolitics (+ cinema)}
		\skill{\faSwimmer}{Swimming, Basket-ball, Cycling}


	\profilesection{Short Bio}
	\aboutme{
    With more than 26 years of experience in web development for digital TV, media platforms, and mobile devices, I spearheaded the architecture, development, and deployment of multiple projects. My leadership focused on improving our software stack through module creation, code reviews, and the adoption of agile practices. Dedicated to crafting user-friendly and scalable solutions leveraging technologies such as AWS, Angular, Vue.js, React, PHP, Python, HTML5, AI.}\\
}
%-------------------------------------------------------------------------------
%                         TABLE ENTRIES RIGHT COLUMN
%-------------------------------------------------------------------------------
\begin{document}


\newpage
\restoregeometry
\sidebarwidth=0.35\paperwidth

\makefrontsidebar

\cvsection{Working Experience}
\begin{cvtable}
    \cvitemRight{Jan,\\2020\\–\\now}
    {Full stack developer (+devops) | Tech lead}
    {difuze, Montreal}
    {
      \textbf{Role}: Primarily similar to the previous mission at LVL Studio\\      \textbf{Project Descriptions}:

      * \textbf{DifuzeGo.com}: Focused on integrating AWS Step Functions in a workflow context, implementing an Express-based node.js worker ecosystem with intercommunicating process nodes and SQLite. \\
      ** \textbf{DifuzeVox.com}: Developed for video description (visual impairment) using Azure Text-to-Speech services, FFMPEG processing, and Auphonic APIs for final mix supervision. Utilized TTML for text and metadata handling.\\
      ** \textbf{AI Detection Project}: Worked on detection of specific mouth symbols (e.g., "pucker" shapes) in films, scene change detection, and burnt-in text detection in video frames using a third-party AI solution. Managed multi-window/browser tab communications and interfacing.\\
      \textbf{Additional Skills and Responsibilities}: Led another web project involving Azure text-to-speech API, served as the main technical referee for 5 developers, and managed multimedia tasks like AWS Media2Cloud solution (transcoding with Mediaconvert, MediaInfo, DRM) and Bitmovin for picture-in-picture and frame accuracy.\\
    }
\end{cvtable}

\begin{cvtable}
    \cvitemRight{Sep,\\2017\\–\\Jan,\\2020}
    {Full stack key player}
    {LVL Studio, Montreal}
    {
      \textbf{Role}: Participated in building a new software stack for our client recently named Difuze, formerly Technicolor Montreal. Involved in architecture decisions, contributed to module development, component creation, including code reviews through GitLab, proactively following the backlog across Jira, arranging/creating/prioritizing tickets, and writing documentation over Confluence.\\
      \textbf{Context}: B2B with www.difuze.com as the customer, www.difuzego.com as the main product.\\
      \textbf{Purpose}: Media Asset Management, screeners (promotional video links), delivery (orders)\\
      \textbf{Tech Areas}: Expertise in Front-end and Back-end development, tasks scheduling (cron jobs), and partially in DevOps.\\
      \textbf{Front-end}: Worked with frameworks and libraries such as Angular 2 (version 7.2.9), RxJS (Observables), Angular Material. Modules including ngx-translate, fortawesome, ipfy, ipstack, ngx-charts, angulartics2, bitmovin player and UI, bowser, fs-symlink, hammerjs, i18n-iso-countries, merge-ranges, moment, ng2-file-upload, ngx-device-detector, ngx-perfect-scrollbar, lodash.merge, rxjs, rxjs-compat, socket.io-client.\\
      \textbf{Back-end}: RESTful web services with OpenAPI, architecture with TypeScript Oriented Architecture (tsoa), modules like exceljs, mailcomposer, moment-timezone, mustache, request, restify, socket.io, user-agent, winston, csv-generate, fs-extra, easy-zip, rapid-api sms, git-repo-info, jasmine, aws-sdk. Asynchronism with Promise and async/await.\\
      \textbf{Package Management}: Proficient with WebPack, Node.js, and yarn.\\
      \textbf{Database and Search Engines}: Managed both new and legacy stacks with technologies like DynamoDb, ElasticSearch, and MySQLWorkbench for stored procedures.\\
      \textbf{AWS DevOps}: Experience with AWS technologies including ElasticBeanStalk, EC2, Lambda, CloudWatch, IAM, aws-cli.\\
      \textbf{Operating Systems}: Ubuntu 18.04, Windows 7, Mac OS Sierra.\\
      \textbf{IDEs}: Efficient with WebStorm, VS Code, and vim / vi.\\
      \textbf{Additional Responsibilities}: Actively participated in maintaining the old stack developed in PHP, based on the YII framework, layered on MySQL as RDS database. Minor developments in C\#.
    }
\end{cvtable}









\newpage
\makebacksidebar



\begin{cvtable}
    \cvitemRight{Jan,\\2017\\–\\Aug,\\2017}
    {Full stack coder}
    {Alayacare, Montreal}
    {
      \textbf{Role}: Development of web features for healthcare applications focusing on front-end and back-end integrations.\\
      * \textbf{Front-end}: Developed pure VUE.js components and a custom time-entry component, leveraging technologies such as super-agent, moment.js, simple-vue-validator, and LESS for enhanced CSS management. \\
      * \textbf{Back-end}: Engineered solutions using PHP 5 with YII Framework and PYTHON for RESTful micro-services, including unit testing and API documentation updates via APIARY. \\
      * \textbf{Mobile and Process Agility}: Performed intensive code reviews for Android and adhered to agile methodologies with three-week sprint cycles, managing tasks through Jira and documentation via Confluence.\\
      * \textbf{Virtualization}: Utilized Docker for encapsulating micro-services and software components, enhancing deployment efficiency and isolation.\\
    }
\end{cvtable}



\begin{cvtable}
    \cvitemRight{May,\\2015\\–\\Dec,\\2016}
    {Front-End Developer}
    {Abbeal, Canal+,  Paris (FR)}
    {
      \textbf{Role}: Development and maintenance of WebApp interfaces for digital TV receivers (STB) across various platforms including TNT, SATELLITE, and IPTV, reaching millions of subscribers.\\
      * \textbf{Front-End}: Managed with WEBPACK, featuring HTML/JS/CSS execution in a Qt/webkit instance on a Linux system. Utilized ES2015 standards with Babel transpilation, GRUNT build management, and NODE.js.\\
      * \textbf{Server-Side}: Operated with UWSGI in Python, hosting the WebApp and handling URL routing via the DBUS interface.\\
      \textbf{Frameworks and Libraries}: Transitioned from BACKBONE to ANGULAR for MVC approaches, used LODASH for data manipulation, EASELJS from CREATEJS suite for canvas operations, and EJS for templating.\\
      \textbf{Testing and Agile Practices}: Conducted unit testing using MOCHA, SPY, SINON; followed agile methodologies with daily standups, grooming sessions, managed tasks and documentation through JIRA and CONFLUENCE. Sources maintained on GitHub.\\
      \textbf{Deployment Achievements}: Successfully deployed to 400,000 TNT receivers and expanded to 2 million SAT receivers.\\
      \textbf{Special Project}: Led the Nightwatch project, a Selenium-centric task-force, to develop UI navigation scenarios, capture and playback network exchanges with REST APIs, managed remotely on AWS EC2 via Docker.\\
    }
\end{cvtable}



\begin{cvtable}
    \cvitemRight{Dec,\\2014\\–\\Apr,\\2015}
    {Android + NodeJS Developer for IoT}
    {Cisco,  Paris (FR)}
    {
      \textbf{Project}: ConnectedLife - Integrated IoT with TV experiences.\\
      * \textbf{Technologies}: Utilized openHAB for IoT ecosystem management, implementing localization with Bluetooth (iBeacons), WiFi, and RFID. Developed a playlist system for delinearized TV based on user analytics, incorporating dynamic ad insertion.\\
      * \textbf{Ambilight \& UI}: Created real-time ambilight effects using GStreamer and Philips Hue via Zigbee. Initiated UI/UX in Dart/WebGL, transitioned to HTML/CSS with Cisco’s JUNO.\\
      * \textbf{Companion Apps}: Crafted apps for wearables and mobiles in Java, enhancing connectivity with WebSocket and Node.js servers.\\
      * \textbf{Achievements}: Showcased at CES 2015, featured at Mobile World Congress, and won "Best IoT Product" award by CSI Magazine.\\
    }
\end{cvtable}



\begin{cvtable}
    \cvitemRight{Aug,\\2014\\–\\Dec,\\2014}
    {HTML5/JS Front-End Developer}
    {Cisco,  Paris (FR)}
    {
      \textbf{Project}: Development of a WebApp for ZON TV Portugal.\\
        Led the design and implementation using Vanilla JS, CSS3, and HTML5, creating a custom UI manager for MVC view arrangement.
        Developed essential components such as channel lists, channel banners, and EPG grids.
        Managed continuous integration with Grunt and Jenkins, oversaw agile processes, and maintained project oversight via JIRA.\\
      \textbf{Outcome}:
        Delivered a fully functional proof of concept that enhanced STB interaction with backend systems, significantly improving user experience.
    }
\end{cvtable}






\newpage
\makethirdsidebar


\begin{cvtable}
    \cvitemRight{Jan,\\2014\\–\\Jul,\\2014}
    {BackboneJS + ThreeJS Front-End Developer}
    {Cisco, Paris (FR)}
    {
      \textbf{Project}: HTML5 app for Belgacom on RDK middleware.\\
      \textbf{Responsibilities}: Developed GPU-accelerated features with CSS3D, WebSockets, and AJAX in BackboneJS; integrated 3D content using Three.JS with animations via Tween.JS; optimized for iOS/Android tablets using Hammer.JS for touch events.\\
      \textbf{Challenges}: Used PhoneGap and CocoonJS for cross-platform packaging, optimizing performance on low-end STBs lacking WebGL.\\
      \textbf{Outcome}: Delivered a high-performance POC with HW acceleration.\\
    }
\end{cvtable}



\begin{cvtable}
    \cvitemRight{Mar,\\2013\\–\\Jan,\\2014}
    {CSS3D + HTML5/JS Front-End Developer}
    {France}
    {
      \textbf{Project}: Developed HTML5, CSS3D applications for embedded devices, showcased at CES 2014.\\
      \textbf{Responsibilities}:
          Built features for devices including LG TVs, HDMI sticks, and tablets using HTML, CSS3D, and WebGL, with BackboneJS for MVC and ThreeJS for 3D modeling.
          Optimized performance via hardware acceleration tests across engines (QT5/Webkit, WebKitNix, Chromium). Managed cross-compilation for platforms like LG TV, Raspberry Pi, and MacOSX.\\
      \textbf{Outcome}: Delivered a high-performance solution, gaining interest from Belgacom and Zon TV.\\
    }
\end{cvtable}


\begin{cvtable}
    \cvitemRight{Dec,\\2010\\–\\Jul,\\2012}
    {WebKit / QT Developer (Full Stack)}
    {NDS Technologies, Paris, France}
    {
      \textbf{Project}: Full-stack development for QT / WebKit integration on specialized chipsets.\\
      \textbf{Responsibilities}:
          Ported and cross-compiled QT / WebKit to target chipsets. Built Java-QT bridges for WebView control, integrating QWebView and Android WebView. Distributed binaries globally via shell scripting (India, Israel). Packaged HTML5 content as .CRX archives for a Cisco France app store.\\
      \textbf{Skills}: Programming, Cross-Compiling, Java-QT Integration, Global Distribution, App Packaging\\
    }
\end{cvtable}
\\
\begin{cvtable}
    \cvitemRight{Nov,\\2006\\–\\Nov,\\2010}
    {Web Engine Integration}
    {Canal +, Paris}
    {
      \textbf{Project}: Integration of the NetFront web engine, coordinating efforts across international teams for seamless deployment.\\
      \textbf{Responsibilities}:
          Led NDS/ACCESS collaborations for middleware integration, managing full-stack dev from standards implementation to cross-platform porting (WIN32, LINUX, chipsets).
          Supervised internships, POCs, including proprietary widget banks and STB-to-iPhone mobile integrations.\\
      \textbf{Skills}: Web Techs, Project Management, X-Departmental Communication\\
    }
\end{cvtable}


\begin{cvtable}
    \cvitemRight{Feb,\\2002\\–\\Oct,\\2006}
    {Full-Stack Browser Integration}
    {Canal+,  Paris (FR)}
    {
      \textbf{Project}: Integration of Opera 6 for Canal+ Technologies.\\
      \textbf{Responsibilities}:
          Spearheaded integration and customization of Opera 6 for Canal+ Technologies, with support for JavaScript 1.3, CSS 2, and Netscape plugins. Led Java-Opera bridge development for advanced multimedia handling and client-specific plugins. Led online game implementations on STB, setting early HTML standards.\\
      \textbf{Skills}: Advanced Web Engine Integration, Cross-Platform Porting, Custom Client-Specific Customization, Multimedia and Game Content Integration.\\
    }
    \cvitemRight{Sep,\\1999\\–\\Dec,\\2001}
    {Pay-TV System Integrator}
    {Canal+,  Paris (FR)}
    {
      \textbf{Project}: System integration and testing for Canal+.\\
      \textbf{Responsibilities}:\
          Developed comprehensive test scripts and managed end-to-end multimedia platform integration.
          Supported complex HW/SW setups providing extensive international client support, both remote and on-site.\\
      \textbf{Skills}: Testing, Hardware/Software Integration, Client Support.\\
    }

    \cvitemRight{Oct,\\1998\\–\\Aug,\\1999}
    {DSP Engineer: AEC for Mobiles}
    {ALTEN, Issy Les Moulineaux (FR)}
    {
      \textbf{Project}: Acoustic Echo Cancellation (AEC) development for improving audio clarity in mobile handsets and speakerphone environments.\\
      \textbf{Responsibilities}:
          Developed and finely tuned AEC algorithms for effective echo suppression, advanced noise reduction, and improved speech quality. Created a Windows executable in Visual C++ to simulate hands-free calls and conduct testing in both real-world and controlled environments.\\
      \textbf{Skills}: Signal Processing, MATLAB, C/C++, Audio Algorithm Tuning.
    }


\end{cvtable}


\end{document}
