% FortySecondsCV LaTeX template
% Copyright © 2019 René Wirnata <rene.wirnata@pandascience.net>
% Licensed under the 3-Clause BSD License. See LICENSE file for details.
%
% Attributions
% ------------
% * fortysecondscv is based on the twentysecondcv class by Carmine Spagnuolo
%   (cspagnuolo@unisa.it), released under the MIT license and available under
%   https://github.com/spagnuolocarmine/TwentySecondsCurriculumVitae-LaTex
% * further attributions are indicated immediately before corresponding code

%-------------------------------------------------------------------------------
%                             ADDITIONAL PACKAGES
%-------------------------------------------------------------------------------
\documentclass[
  a4paper,
%   showframes,
   maincolor=cvblue,
   sectioncolor=cvblue,
%  subsectioncolor  =orange
%   sidebarwidth=0.4\paperwidth,
%   topbottommargin=0.03\paperheight,
%   leftrightmargin=20pt
]{fortysecondscv}

% improve word spacing and hyphenation
\usepackage{microtype}
\usepackage{ragged2e}
\usepackage{rotating}

% take care of proper font encoding
\ifxetex
	\usepackage{fontspec}
	\defaultfontfeatures{Ligatures=TeX}
% \newfontfamily\headingfont[Path = fonts/]{segoeuib.ttf} % local font
\else
	\usepackage[utf8]{inputenc}
	\usepackage[T1]{fontenc}
% \usepackage[sfdefault]{noto} % use noto google font
\fi

% enable mathematical syntax for some symbols like \varnothing
\usepackage{amssymb}

% bubble diagram configuration
\usepackage{smartdiagram}
\smartdiagramset{
  % defaut font size is \large, so adjust to harmonize with sidebar layout
  bubble center node font = \footnotesize,
  bubble node font = \footnotesize,
  % default: 4cm/2.5cm; make minimum diameter relative to sidebar size
  bubble center node size = 0.3\sidebartextwidth,
  bubble node size = 0.25\sidebartextwidth,
  distance center/other bubbles = 1.5em,
  % set center bubble color
  bubble center node color = maincolor!70,
  % define the list of colors usable in the diagram
  set color list = {maincolor!10, maincolor!40,
  maincolor!20, maincolor!60, maincolor!35},
  % sets the opacity at which the bubbles are shown
  bubble fill opacity = 0.8,
}


%-------------------------------------------------------------------------------
%                            PERSONAL INFORMATION
%-------------------------------------------------------------------------------
% profile picture
\cvprofilepic{pics/profile}
% your name
\cvname{\LARGE Hugo MACHEFER}
% job title/career
\cvjobtitle{Full Stack Developer | Devops | Tech Lead }

% phone number
\cvphone{001 514 659 0718}
% email address
\cvmail{hugo.machefer@gmail.com}

\cvlinkedin{linkedin.com/in/hugomachefer}{hugomachefer}

\cvgithub{https://github.com/hmachefe}{hugomachefer}

% short address/location, use \newline if more than 1 line is required
\cvaddress{5563 avenue de Darlington, H3T1T1 Montréal}


% add additional information
% \newcommand{\additional}{some more?}


% add more profile sections to sidebar on first page
\addtofrontsidebar{
	% include gosquare national flags from https://github.com/gosquared/flags;
	% naming according to ISO 3166-1 alpha-2 country codes
	\graphicspath{{pics/flags/}}

	\profilesection{Skills}
    \chartlabel{Programmatic languages}\\
      \pointskillNinety{}{TypeScript, JavaScript}
      \pointskillNinety{}{HTML (HTML 5), CSS (3D)}
      \pointskillEighty{}{PHP (v7 et v5)}
      \pointskillSeventy{}{Python (v2 et v3)}
      \pointskillSixty{}{C\# (.NET, Mono)}
      \pointskillFifty{}{Java (Fx, Swing, Maven)}
      \pointskillFifty{}{Matlab}
      \pointskillForty{}{QT (4 and 5)}
    \chartlabel{Front-end (web)}\\
      \pointskillNinety{}{Angular (9/10/14)}
      \pointskillEighty{}{jQuery}
      \pointskillSeventy{}{React}
      \pointskillSeventy{}{RxJS / Observables}
      \pointskillSeventy{}{SCSS / SASS}
      \pointskillSeventy{}{Bootstrap (Responsive)}
      \pointskillSixty{}{Backbone}
      \pointskillFifty{}{VueJS}
      \pointskillForty{}{ThreeJS (3D)}
      \pointskillThirty{}{LaTeX (Typesetting)}
    \chartlabel{Back-end}\\
      \pointskillNinety{}{OpenAPI/Swagger (TSOA)}
      \pointskillSeventy{}{Express (Node.js)}
      \pointskillSixty{}{Microservices (in Python)}
      \pointskillSixty{}{Yii (PHP framework)}
      \pointskillThirty{}{Django (Python framework)}
      \pointskillThirty{}{Elixir (Robust scalability)}
      \pointskillThirty{}{Knex (Postgres, SQLite)}
    \chartlabel{Database}\\
      \pointskillNinety{}{DynamoDB / (MongoDB)}
      \pointskillEighty{}{ElasticSearch}
      \pointskillSeventy{}{PostGres, Sqlite, MySql}
      \pointskillForty{}{Microsoft SQL Server}
      \pointskillSeventy{}{MySQLWorkbench, DBeaver}


}


%-------------------------------------------------------------------------------
%                              SIDEBAR 2nd PAGE
%-------------------------------------------------------------------------------
\addtobacksidebar{
\vspace*{-6mm}  % Adjusts the vertical space, move content up by 10mm
\chartlabel{AWS Cloud}\\
\pointskillNinety{}{EC2, S3, CloudWatch}
\pointskillNinety{}{Lambdas (layer)}
\pointskillNinety{}{Secrets Manager}
\pointskillEighty{}{IAM, ECR, SES, SNS}
\pointskillEighty{}{CloudFront (CDN)}
\pointskillSeventy{}{Elastic Beanstalk}
\pointskillSeventy{}{Step Functions}
\pointskillTwenty{}{Fargate (ECS)}
\pointskillEighty{}{Route53 (DNS)}
\pointskillEighty{}{Certificates (ACM)}

\chartlabel{Azure Cloud}\\
\pointskillForty{}{Text To Speech Service}

\chartlabel{DevOps}\\
\pointskillEighty{}{Terraform (IaC Multi-Cloud)}
\pointskillSeventy{}{Kubernetes (Orchestration)}
\pointskillFifty{}{CloudFormation (IaC AWS)}

\chartlabel{HTML5 players}\\
\pointskillNinety{}{Bitmovin}
\pointskillNinety{}{Accurate Player (Codemill)}
\pointskillSeventy{}{Video.js (open source)}

\chartlabel{Multimedia}\\
\pointskillNinety{}{FFMPEG, mediainfo}
\pointskillNinety{}{AWS MediaConvert}
\pointskillSeventy{}{Hybrik (Dolby)}

\chartlabel{Mobile Development}\\
\pointskillSixty{}{Java Android (Dalvik)}
\pointskillForty{}{Ionic (Apache Cordova)}
\pointskillThirty{}{Xcode (Objective C)}


\chartlabel{Version Control Systems}\\
\pointskillEighty{}{Git (gitlab, github)}
\pointskillForty{}{SVN / Synergy}

\chartlabel{Operating Systems}\\
\pointskillEighty{}{Linux (Ubuntu 22)}
\pointskillSeventy{}{Windows (Win 10)}
\pointskillFifty{}{Mac (El Capitan 10.11.2)}
\pointskillSixty{}{Tizen (Samsung TV)}
\pointskillFifty{}{WebOS (LG TV)}

\chartlabel{IDE}\\
\pointskillEighty{}{Webstorm (Jetbrains)}
\pointskillEighty{}{Visual Code Editor}
\pointskillSeventy{}{Vim / Vi}


\chartlabel{Agility}\\
\pointskillEighty{}{Jira (backlog, sprint)}
\pointskillEighty{}{Confluence / wiki}
\pointskillEighty{}{Scrum (grooming, review)}

}

\addtothirdsidebar{
\vspace*{-9mm}  % Adjusts the vertical space, move content up by 10mm

  \chartlabel{Scripting}\\
  \pointskillEighty{}{Bash, shell (Linux, Mac)}
  \pointskillSeventy{}{CMD prompt, Powershell}

  \chartlabel{Streaming, DRM encryption}\\
  \pointskillEighty{}{MPEG/DASH, Widevine}
  \pointskillEighty{}{HLS, AES-128}

  \chartlabel{Instant messaging}\\
  \pointskillEighty{}{Teams, Slack}

  \chartlabel{UX / UI}\\
  \pointskillSeventy{}{Figma, Miro}

  \chartlabel{AI}\\
  \pointskillEighty{}{Chatgpt 4.0}
  \pointskillSixty{}{OpenAI API}

	\profilesection{Languages}
  \vspace{-1mm} % Adjust the negative space as needed
	\barskill{}{\textbf{French} (Mother Tongue)}{100}
	\barskill{}{\textbf{English} (TOEFL)}{92}
	\barskill{}{\textbf{German} (Educational, 2 years)}{35}
	\barskill{}{\textbf{Spanish} (Beginner, Duolingo)}{25}

	\profilesection{Honors \& awards}
    \vspace{-1mm} % Adjust the negative space as needed
		\skill{\faAward}{NDS CEO Award · Jan 2006}
		\skill{\faAward}{CSI Awards 2014}

  \profilesection{Teaching, Talks }
    \vspace{-1mm} % Adjust the negative space as needed
		\skill{\faChalkboardTeacher}{HTML5 coach (India, Israel, UK, US)}
    \skill{\faChalkboardTeacher}{Cross-cultural trainer, TMC}
		\skill{\faChalkboardTeacher}{ WebRTC (Intl dev conference, Berlin)}

  \vspace{-3mm} % Adjust the negative space as needed
	\profilesection{Education}
    \vspace{-1mm} % Adjust the negative space as needed
		\skill{\faLaptop}{Engineer's degree (ESEO) 1998}

	\profilesection{Interests}
    \vspace{-1mm} % Adjust the negative space as needed
		\skill{\faUnity}{WebGl/VR/AR, streaming, AI}
		\skill{\faGlobe}{International geopolitics (+ cinema)}
		\skill{\faSwimmer}{Swimming, Basket-ball, Cycling}


	\profilesection{Short Bio}
  \vspace{-1mm} % Adjust the negative space as needed
	\aboutme{
    With 26+ years in web development, I have led multiple projects across digital TV and media platforms, enhancing our software stack with agile practices and technologies like AWS, Angular, Vue.js, React, PHP, Python, HTML5, and AI
    }\\

}
%-------------------------------------------------------------------------------
%                         TABLE ENTRIES RIGHT COLUMN
%-------------------------------------------------------------------------------
\begin{document}


\newpage
\restoregeometry
\sidebarwidth=0.35\paperwidth

\makefrontsidebar

\vspace*{-2.9em} % Adjust the value as needed
\cvsection{Working Experience}
\begin{cvtable}
    \cvitemRight{Jan,\\2020\\–\\now}
    {Full stack developer (+devops) | Tech lead}
    {difuze, Montreal}
    {
      \textbf{Role}: Primarily similar to the previous mission at LVL Studio\\
        \noindent\hspace*{1mm}*\textbf{difuzego.com:} Enhanced the initial solution.
        Migration to Angular 14.
        I introduced the MFA authenticator.
        Part of our data was moved from ElasticSearch into Postgre, whose model became knex-compliant.
        Also a file publishing (upload) module has been implemented,
        ingested in a workflow system based on AWS Step Functions,
        producing streamable proxies through compression.
        This workflow solution was further expanded to include an on-prem \textit{workers}
        ecosystem, Express/Node.js-centric featuring nodes child process intercommunication and SQLite.
        Additionally, I contributed to the project's streaming and encryption tasks,
        including MPEG/DASH encoding on AWS by Mediaconvert (at start but replaced later by Hybrik) and encrypted by KeyOS/BuyDRM;
        including HLS encoding performed on AWS EC2 MacOS (at start but replaced later by local FFMPEG processing), encrypted using the AES-128 algorithm.
        In this project, Postgres is used, and µsoft SQL Server over tedious; HTML5 video players are Bitmovin and Accurate \\
        \noindent\hspace*{1mm}*\textbf{difuzevox.com}: For the video description project aiding visual impairment, I managed the integration of Azure Text-to-Speech, local FFMPEG, and MediaInfo with Auphonic APIs. We used TTML for voice generation and Video.js for multimedia rendering, all under Terraform management. I supervised three developers and one QA, focusing on enhancing media accessibility.\\
        \noindent\hspace*{1mm}*\textbf{AI Detection Project}: Focused on the detection of specific mouth symbols (e.g., "pucker" shapes) in video frames, scene change detection, and burnt-in text detection using a 3rd-party AI solution. Managed multi-window/browser tab communications via postMessage() and interfacing.\\
    \textbf{CI/CD}: I participated in a small project deployed using Kubernetes.\\
    \textbf {npm modules}: axios, knex, objection, pg, qrcode, node-ec2-metadata, nodemon, speakeasy, srt2vtt, unirest, uuid-validate, git-repo-info, jest-preset-angular, jest-spec-reporter, mysql-json, puppeteer, hash-wasm, ngx-device-detector, videojs-playlist, videojs-wavesurfer, fast-diff, ngx-virtual-scroller, angular2-text-mask, js-levenshtein, http-proxy, elementtree, chalk, bcrypt...
  }
\end{cvtable}


\begin{cvtable}
    \cvitemRight{Sep,\\2017\\–\\Jan,\\2020}
    {Full-Stack coder, Media Asset Management}
    {LVL Studio, Montreal}
    {
      \textbf{Role}: As a key contributor to Difuze’s new software stack, I was responsible for architectural decisions, module development, and component creation. I proactively managed the backlog using Jira, conducted code reviews via GitLab, and maintained detailed documentation in Confluence. In a B2B environment with Difuze Inc. as our primary client, my focus was on enhancing difuzego.com, our main product.\\
         ** \noindent\hspace*{1mm}\textbf{showcase.com}: Old legacy stack based on YII written in PHP, along with scripts in C\# for scheduling tasks like cron jobs\\
         ** \noindent\hspace*{1mm}\textbf{difuzego.com}: New stack based on \textit{modern} AWS features written in TS.\\
      \textbf{Purpose}: Development based on a Media Asset Management (MAM) system accessed through HTTPS requests using SOAP protocol, enabling content display in catalog form, with the ability to share films as screeners, preceding deliveries (orders).\\
      \textbf{Front-end}: Angular 2 (v7.2.9), RxJS, Angular Material, also including various NPM modules like ngx-translate, bitmovin, fs-symlink, hammerjs, ngx-perfect-scrollbar, fortawesome, xml2js, subtitle-converter, socket.io-client, react-image-file-resizer, ng2-file-upload, moment-timezone, i18n-iso-countries, d3, bowser, jsoneditor, lodash, jschardet...\\
      \textbf{Back-end}: RESTful services with OpenAPI, TypeScript Oriented Architecture (tsoa), and NPM imports like exceljs, fs-extra, mustache, socket.io, winston-console-formatter, winston-cloudwatch, uuid-validate, useragent, restify-cors-middleware, restify-errors, reflect-metadata, mime-types, jsonwebtoken, form-data, camelcase, aws-sdk ...; async/await.\\
      \textbf{Package Management}: Proficient with WebPack, Node.js, and yarn.\\
      \textbf{Database and Search}: Managed both legacy (MySQL RDS) and new stack (DynamoDb, ElasticSearch).\\
      \textbf{AWS DevOps}: Experience with ElasticBeanStalk, EC2, S3, small Lambda, CloudWatch, IAM, ECR. Infra deployment automated with Cloudformation\\
      \textbf{Additional Work}: Maintained legacy PHP/YII stack and MySQL RDS; minor C\# developments supervised over Mono on Ubuntu\\
      \textbf{Architecture}: Monolithic, no µservices. Shared data model UI \& API\\
      \textbf{CI/CD}: pipelines over gitlab including linter, unitary tests, builder, deployer
    }
\end{cvtable}







\newpage
\makebacksidebar



\begin{cvtable}
    \cvitemRight{Jan,\\2017\\–\\June,\\2017}
    {Full Stack Developer in Healthcare Sector}
    {AlayaCare, Montreal}
    {
      \textbf{Role}: Developed web functionalities for healthcare platforms, focusing on both front-end and back-end.\\
      * \textbf{Front-end}: Created VUE.js components without jQuery, using libraries such as super-agent for AJAX requests, moment.js for date formatting, simple-vue-validator for validation, and LESS for optimized CSS. Built a custom time-entry component based on jquery.timeentry outside of VUE.js.\\
      * \textbf{Back-end}: Implemented PHP 5 solutions within the YII Framework, integrating the Carbon extension for enhanced date management, along with Python micro-services using SqlAlchemy. Ensured quality with precise unit tests and maintained accurate REST API documentation via APIARY.\\
      * \textbf{Mobile and Process Agility}: Conducted detailed Android code reviews and operated within a three-week sprint cycle, actively participating in standups, demos, retros, and grooming. Managed tasks and backlogs through Jira, with documentation centralized in Confluence.\\
      * \textbf{Virtualization}: Deployed each micro-service and software component in Docker containers, facilitating isolated, efficient deployment.\\
    }
\end{cvtable}




\begin{cvtable}
    \cvitemRight{May,\\2015\\–\\Dec,\\2016}
    {Front-End Developer in the Digital TV Sector}
    {Abbeal for Canal+, FR}
    {
      \textbf{Role}: Developed and maintained WebApp interfaces for Canal+ digital TV receivers, reaching millions on TNT, SATELLITE, and IPTV.\\
      * \textbf{Front-End}: Engineered HTML, JavaScript, and CSS solutions, managed by WEBPACK, and executed in a Qt/WebKit instance on a Linux system. Utilized ES2015 standards via Babel, with builds managed by GRUNT and NODE.js.\\
      * \textbf{Server-Side}: Supported by a Python UWSGI server hosting the WebApp and managing routing through DBUS for system interactions.\\
      * \textbf{Libraries and Frameworks}: Transitioned from BACKBONE to ANGULAR for enhanced MVC architecture efficiency; utilized LODASH for advanced data manipulation, EASELJS for robust interactive canvas operations, and EJS for sophisticated and efficient templating solutions.\\
      * \textbf{Agile and Testing}: Conducted unit tests with MOCHA, SPY, and SINON, integrating POSTMAN for API testing. Managed agile sprints, tracking tasks on JIRA and documenting in CONFLUENCE.\\
      * \textbf{Deployment}: Successfully deployed to 400,000 TNT receivers and subsequently expanded to over 2 million SAT subscribers globally.\\
      * \textbf{Nightwatch Project}: Led a Selenium-based task for UI navigation testing and REST API simulation, implemented on AWS EC2 with Docker.\\
    }
\end{cvtable}






\begin{cvtable}
    \cvitemRight{Dec,\\2014\\–\\Apr,\\2015}
    {Android + NodeJS Developer for IoT}
    {Cisco, Paris (FR)}
    {
      \textbf{Project}: ConnectedLife - Seamlessly integrated IoT technology into advanced TV experiences for sophisticated home automation systems.\\
      * \textbf{Technologies}: Leveraged openHAB to orchestrate IoT, using Bluetooth (iBeacons), WiFi triangulation (Cisco MSE/Prime), and RFID (Impinj) for localization. Developed a delinearized TV channel experience with personalized playlists and targeted ad insertion using Cisco analytics.\\
      * \textbf{Ambilight \& UI}: Designed real-time ambilight effects by processing TV frames with GStreamer to match room lighting (Philips Hue) effectively through Zigbee. Initially developed UI in Dart/WebGL and smoothly transitioned to HTML/CSS using Cisco’s proprietary JUNO framework.\\
      * \textbf{Companion Apps}: Built companion apps for wearables and Samsung mobiles in Java, ensuring robust, seamless real-time connectivity through WebSocket and high-performance, dedicated Node.js servers.\\
      * \textbf{Achievements}: Demonstrated at CES 2015, prominently featured at Mobile World Congress, and awarded \textit{Best IoT Product} by CSI Magazine.\\
    }
\end{cvtable}







\begin{cvtable}
    \cvitemRight{Aug,\\2014\\–\\Dec,\\2014}
    {HTML5/JS Front-End Developer}
    {Cisco, Paris (FR)}
    {
      \textbf{Project}: Proof of concept WebApp for ZON TV Portugal.\\
      * \textbf{Technologies}: Developed using Vanilla JS, CSS3, and HTML5, with a custom UI manager to structure MVC views without a framework. Built core components such as the channel list, channel banner, and EPG grid.\\
      * \textbf{Integration and Workflow}: Utilized QT on RDK middleware for optimized STB-to-backend data handling. Employed Grunt (Node) and Jenkins for continuous integration, managing tasks through JIRA, and collaborating in scrum sessions.\\
      * \textbf{Outcome}: Successfully delivered a high-performance WebApp that improved interaction between STB and ZON TV backend, showcased as a proof of concept.
    }
\end{cvtable}







\newpage
\makethirdsidebar

\vspace*{-2em} % Adjust the value as needed
\begin{cvtable}
    \cvitemRight{Jan,\\2014\\–\\Jul,\\2014}
    {BackboneJS + ThreeJS Front-End Developer}
    {Cisco, Paris (FR)}
    {
      \textbf{Project}: HTML5 app for Belgacom on RDK middleware.\\
      \textbf{Responsibilities}: Developed GPU-accelerated features with CSS3D, WebSockets, AJAX, and Three.JS in BackboneJS; optimized for tablets with Hammer.JS for touch events.
      Packaged via PhoneGap, optimized for low-end STBs without WebGL.\\
      \textbf{Outcome}: Delivered high-performance POC with HW acceleration.\\
    }
\end{cvtable}




\begin{cvtable}
    \cvitemRight{Mar,\\2013\\–\\Jan,\\2014}
    {CSS3D + HTML5/JS Front-End Developer}
    {NDS Technologies, Paris}
    {
      \textbf{Project}: Advanced HTML5 development for embedded devices in digital TV (smart TVs, HDMI sticks).\\
      \textbf{Responsibilities}:
          Built features for devices including LG TVs, HDMI sticks, and tablets using HTML, CSS3D, and WebGL, with BackboneJS for MVC and ThreeJS for 3D modeling.
          Optimized performance via hardware acceleration tests across engines (QT5/Webkit, WebKitNix, Chromium). Managed cross-compilation for platforms like LG TV, Raspberry Pi, and MacOSX.\\
      \textbf{Outcome}: High-performance solution, attracting Belgacom and Zon TV.\\
    }
\end{cvtable}


\begin{cvtable}
    \cvitemRight{Aug,\\2012\\–\\Feb,\\2013}
    {AngularJS | BackboneJS Front-End Developer}
    {Cisco (ex NDS), FR}
    {
      \textbf{Project}: HTML5 software solution for digital TV with patented remote UI and second-screen functionality.\\
      \textbf{Responsibilities}:
          Contributed to a patented HTML5 rendering solution for payTV, showcased at IBC. Assessed frameworks (Angular, Backbone, Spine) for HTML5 EPG on STB, finalizing Backbone. Built MVC architecture for set-top box apps and explored FLASH-to-HTML5 tools like Adobe Edge.\\
      \textbf{Skills}: HTML5 Development, Framework Evaluation, MVC Architecture, Open-Source Tools (GIT, Jira, Jenkins)\\
    }
\end{cvtable}


\begin{cvtable}
    \cvitemRight{Dec,\\2010\\–\\Jul,\\2012}
    {WebKit / QT Developer (Full Stack)}
    {NDS Technologies, Paris, France}
    {
      \textbf{Project}: Comprehensive full-stack development for QT / WebKit integration on specialized chipsets, with a focus on performance and compatibility.\\
      \textbf{Responsibilities}:
          Cross-compiled QT / WebKit for various chipsets. Created Java-QT bridges for WebView control (QWebView and Android WebView). Automated binary distribution with shell scripting (India, Israel). Packaged HTML5 content as .CRX archives for Cisco France’s app store.\\
      \textbf{Skills}: Advanced Programming, Efficient Cross-Compiling, Java-QT Integration, Global Distribution Automation, Secure App Packaging\\
    }
\end{cvtable}
\\
\begin{cvtable}
    \cvitemRight{Nov,\\2006\\–\\Nov,\\2010}
    {Web Engine Integration}
    {Canal+, Paris (FR)}
    {
      \textbf{Project}: Integrated NetFront web engine with global team coordination.\\
      \textbf{Responsibilities}:
          Led NDS/ACCESS collaboration on middleware integration, managing full-stack development and X-platform porting; supervised internships, POCs, and widget bank for STB-to-iPhone integration.\\
      \textbf{Skills}: Web Tech, Project Management, X-Dept. communication, C coding\\
    }
\end{cvtable}



\begin{cvtable}
    \cvitemRight{Feb,\\2002\\–\\Oct,\\2006}
    {Full-Stack Browser Integration}
    {Canal+, Paris (FR)}
    {
      \textbf{Project}: Integrated Opera 6 into Canal+ Technologies.\\
      \textbf{Responsibilities}:
          Led Opera 6 integration with JavaScript 1.3, CSS 2, and Netscape plugins. Built Java-Opera bridges for multimedia and custom plugins. Managed STB games, setting HTML standards.\\
      \textbf{Skills}: Web Engine Integration, Cross-Platform Porting, Client Customization, Multimedia/Game Integration. C++ programming\\
    }

    \cvitemRight{Sep,\\1999\\–\\Dec,\\2001}
    {Pay-TV System Integrator}
    {Canal+,  Paris (FR)}
    {
      \textbf{Project}: System integration and testing for Canal+.\\
      \textbf{Responsibilities}:\
          Developed comprehensive test scripts and managed end-to-end multimedia platform integration.
          Supported complex HW/SW setups providing extensive international client support, both remote and on-site.\\
      \textbf{Skills}: Testing, Hardware/Software Integration, Client Support.\\
    }

    \cvitemRight{Oct,\\1998\\–\\Aug,\\1999}
    {DSP Engineer: AEC for Mobiles}
    {ALTEN, Issy Les Moulineaux (FR)}
    {
      \textbf{Project}: Developed Acoustic Echo Cancellation (AEC) to enhance audio clarity in mobile devices.\\
      \textbf{Responsibilities}:
          Tuned AEC algorithms for echo suppression, noise reduction, and speech quality. Built a Visual C++ executable for hands-free call simulation and testing in varied environments.\\
      \textbf{Skills}: Signal Processing, MATLAB, C/C++, Audio Tuning.
    }
\end{cvtable}



\end{document}
